%!TEX TS-program = xelatex
\documentclass{sener2025}

\usepackage{lipsum}

% ============================================================================
% INFORMACIÓN DEL DOCUMENTO
% ============================================================================
\title{Guía de Citas y Referencias}
\subtitle{Formato APA para documentos oficiales}
\author{Unidad de Planeación Energética}
\date{Noviembre 2025}
\institucion{Secretaría de Energía}
\unidad{Unidad de Planeación Energética}
\setDocumentoCorto{Guía de Citas APA}
\palabrasclave{bibliografía, APA, citas, referencias}

\begin{document}

\maketitle

\tableofcontents
\clearpage

\section{Introducción a las Citas en Formato APA}

Este documento muestra ejemplos de cómo citar diferentes tipos de fuentes usando el formato APA (American Psychological Association) en documentos oficiales de la Secretaría de Energía.

\begin{notaimportante}
Todas las referencias citadas en el texto deben aparecer en la sección de Referencias Bibliográficas al final del documento, y viceversa.
\end{notaimportante}

\section{Tipos de Citas}

\subsection{Citas en paréntesis (narrativas)}

Las citas en paréntesis se usan cuando la referencia no forma parte de la estructura de la oración:

\begin{ejemplo}
El sector eléctrico mexicano ha experimentado transformaciones significativas en los últimos años \autocite{sener2024pladese}. Las energías renovables muestran un crecimiento sostenido \autocite{gomez2023renovables}.
\end{ejemplo}

\subsection{Citas integradas en el texto}

Las citas integradas se usan cuando el autor forma parte de la oración:

\begin{ejemplo}
Como señalan \textcite{rodriguez2023planeacion}, las metodologías de planeación energética requieren una visión integral del sistema. Según \textcite{hernandez2024solar}, la eficiencia de los sistemas fotovoltaicos varía significativamente según la región.
\end{ejemplo}

\subsection{Citas múltiples}

Para citar varias fuentes simultáneamente:

\begin{ejemplo}
Diversos estudios han documentado el crecimiento del sector renovable en México \autocite{gomez2023renovables,hernandez2024solar,lopez2023hidro}.
\end{ejemplo}

\section{Ejemplos por Tipo de Fuente}

\subsection{Libros}

\subsubsection{Libro completo}

Los libros de texto especializados son fundamentales para la formación técnica. \textcite{rodriguez2023planeacion} presentan una metodología integral para la planeación energética. En el contexto internacional, \textcite{smith2022energy} ofrecen un análisis detallado de sistemas renovables.

\subsubsection{Capítulo de libro}

El potencial hidroeléctrico del sureste mexicano ha sido ampliamente documentado \autocite{lopez2023hidro}, identificando oportunidades de desarrollo en múltiples cuencas hidrológicas.

\subsection{Artículos de Revista}

\subsubsection{Artículo con DOI}

Las energías renovables han mostrado un crecimiento del 12\% anual en el periodo 2020-2023 \autocite{gomez2023renovables}. Este crecimiento se concentra principalmente en tecnologías solar y eólica.

\subsubsection{Artículo sin DOI}

Los sistemas fotovoltaicos en zonas áridas presentan eficiencias superiores al promedio nacional \autocite{hernandez2024solar}, alcanzando factores de planta de hasta 25\%.

\subsection{Reportes e Informes Técnicos}

\subsubsection{Reportes gubernamentales}

El \textcite{sener2024pladese} establece las directrices para el desarrollo del sector eléctrico hasta 2038. El \textcite{cenace2024transmision} documenta la operación del sistema durante 2024.

\subsubsection{Reportes de organismos reguladores}

La \textcite{cre2023regulacion} define el marco normativo aplicable al sector eléctrico mexicano, estableciendo los criterios para la participación de generadores privados.

\subsection{Páginas Web y Recursos en Línea}

\subsubsection{Sitios web institucionales}

Para información actualizada sobre el sector energético, consulte el portal oficial \autocite{sener2024portal}. Las estadísticas oficiales están disponibles en \autocite{sie2024datos}.

\subsubsection{Reportes en línea de organismos internacionales}

La \textcite{iea2023outlook} proyecta un escenario de transición energética acelerada a nivel global, con implicaciones importantes para México.

\subsection{Tesis}

\subsubsection{Tesis doctoral}

\textcite{sanchez2023optimizacion} desarrolló algoritmos genéticos para la optimización de redes de transmisión, logrando reducciones del 15\% en pérdidas técnicas.

\subsubsection{Tesis de maestría}

El potencial eólico del Istmo de Tehuantepec supera los 10,000 MW según \textcite{torres2024eolica}, quien realizó mediciones durante 24 meses.

\subsection{Conferencias}

Las redes inteligentes representan el futuro de la distribución eléctrica \autocite{morales2023smart}, permitiendo una gestión más eficiente de la demanda.

\subsection{Documentos Legales y Normativos}

\subsubsection{Leyes}

El marco legal del sector eléctrico está definido por la \textcite{ley2014energia}, que establece las bases para la participación de los sectores público y privado.

\subsubsection{Normas oficiales}

Las instalaciones eléctricas deben cumplir con la \textcite{nom2018eficiencia}, que establece los requisitos mínimos de seguridad.

\subsection{Videos y Multimedia}

La Secretaría de Energía ha producido material audiovisual sobre la transición energética \autocite{sener2023video}, disponible en plataformas digitales.

\subsection{Bases de Datos}

Los indicadores de desarrollo energético a nivel internacional están disponibles en \autocite{worldbank2024data}, permitiendo comparaciones entre países.

\section{Tabla Resumen de Comandos de Citas}

\begin{tablaguinda}
  \caption{Comandos de citación en LaTeX con biblatex-apa}
  \label{tab:comandos-citas}
  \begin{tabularx}{\textwidth}{lXl}
    \toprule
    \encabezadoguinda{Comando} & \encabezadoguinda{Descripción} & \encabezadoguinda{Ejemplo} \\
    \midrule
    \texttt{\textbackslash autocite\{\}} & Cita en paréntesis & (SENER, 2024) \\
    \texttt{\textbackslash textcite\{\}} & Cita integrada en texto & SENER (2024) \\
    \texttt{\textbackslash parencite\{\}} & Cita entre paréntesis & (SENER, 2024) \\
    \texttt{\textbackslash cite\{\}} & Cita básica & SENER, 2024 \\
    \texttt{\textbackslash citeauthor\{\}} & Solo autor & SENER \\
    \texttt{\textbackslash citeyear\{\}} & Solo año & 2024 \\
    \texttt{\textbackslash citetitle\{\}} & Solo título & Plan de Desarrollo... \\
    \bottomrule
  \end{tabularx}
\end{tablaguinda}

\section{Buenas Prácticas}

\begin{datosclave}
\textbf{Recomendaciones para citas en documentos oficiales:}
\begin{itemize}
  \item Usar \texttt{\textbackslash autocite} para citas generales
  \item Usar \texttt{\textbackslash textcite} cuando el autor es parte de la oración
  \item Verificar que todas las citas tengan su entrada en referencias.bib
  \item Incluir DOI cuando esté disponible
  \item Para URLs, incluir fecha de consulta con \texttt{urldate}
  \item Usar nombres completos de instituciones entre llaves dobles: \texttt{\{\{SENER\}\}}
\end{itemize}
\end{datosclave}

\section{Compilación del Documento}

Para que las citas y referencias se generen correctamente, compile en este orden:

\begin{recuadro}
\textbf{Secuencia de compilación:}
\begin{enumerate}
  \item \texttt{xelatex documento.tex}
  \item \texttt{biber documento}
  \item \texttt{xelatex documento.tex}
  \item \texttt{xelatex documento.tex}
\end{enumerate}

O usar latexmk para automatizar:
\begin{verbatim}
latexmk -xelatex documento.tex
\end{verbatim}
\end{recuadro}

\section{Ejemplo Completo de Párrafo con Múltiples Citas}

El desarrollo del sector eléctrico nacional requiere una planeación integral que considere aspectos técnicos, económicos y ambientales \autocite{rodriguez2023planeacion}. En este contexto, el \textcite{sener2024pladese} establece metas ambiciosas para la incorporación de energías limpias, alcanzando el 35\% de la capacidad instalada para 2030.

Las energías renovables han mostrado un crecimiento sostenido \autocite{gomez2023renovables}, particularmente en tecnologías solar y eólica. \textcite{hernandez2024solar} documentan eficiencias superiores al 20\% en sistemas fotovoltaicos instalados en zonas áridas, mientras que \textcite{torres2024eolica} identifica un potencial eólico superior a 10,000 MW en el Istmo de Tehuantepec.

La operación del sistema eléctrico nacional presenta desafíos importantes \autocite{cenace2024transmision}, especialmente en términos de integración de fuentes variables. \textcite{sanchez2023optimizacion} propone algoritmos de optimización que podrían reducir las pérdidas técnicas en un 15\%.

A nivel internacional, la \textcite{iea2023outlook} proyecta una transición energética acelerada, con implicaciones significativas para países en desarrollo como México. Estas proyecciones son consistentes con los datos del \textcite{worldbank2024data}, que muestran una tendencia global hacia la descarbonización del sector eléctrico.

% ============================================================================
% BIBLIOGRAFÍA
% ============================================================================

\clearpage
\printbibliography[title={Referencias Bibliográficas}]

\end{document}

%!TEX TS-program = xelatex
\documentclass{sener2025}

% Utilidades
\usepackage{lipsum}
\usepackage{float}

% ============================================================================
% INFORMACIÓN DEL DOCUMENTO
% ============================================================================
\title{Programa de Desarrollo del Sistema Eléctrico Nacional}
\subtitle{Nueva plantilla de comunicación}
\author{Subsecretaría de Planeación y Transición Energética}
\date{Noviembre 2025}
\institucion{Secretaría de Energía}
\unidad{Subsecretaría de Planeación y Transición Energética}
\setDocumentoCorto{PRODESEN 2025–2039}
\palabrasclave{energía, planeación, sistema eléctrico, renovables}
\version{2.0}

\begin{document}

% ============================================================================
% PORTADA CON FONDO GUINDA
% ============================================================================
\portadafondo

% ============================================================================
% PÁGINA DE CRÉDITOS/DIRECTORIO
% ============================================================================
\paginacreditos{
  \vspace{1cm}
  
  \textbf{Secretaria de Energía}\\
  Lic. Luz Elena González Escobar\\[0.5cm]
  
  \textbf{Subsecretario de Planeación y Transición Energética}\\
  Ing. Leonardo Beltrán Rodríguez\\[0.5cm]
  
  \textbf{Director General de Planeación e Información Energéticas}\\
  Dr. Edmundo Gil Borja\\[0.5cm]
  
  \textbf{Unidad de Planeación Energética}\\
  Equipo técnico de elaboración\\[1cm]
  
  \hrule
  \vspace{0.5cm}
  
  \textit{Primera edición: Noviembre 2025}\\
  \textit{Distribución gratuita. Prohibida su venta.}
}

% ============================================================================
% TABLA DE CONTENIDOS E ÍNDICES
% ============================================================================
\tableofcontents
\clearpage

\listatablas
\listafiguras

% ============================================================================
% RESUMEN EJECUTIVO
% ============================================================================

\begin{resumenejecutivo}
Este documento presenta la \textbf{plantilla institucional completa} de la Secretaría de Energía, diseñada para la elaboración de documentos oficiales con identidad gráfica institucional.

\textbf{Funcionalidades incluidas:}
\begin{itemize}
  \item Portada con fondo institucional o estándar
  \item \textbf{NUEVO:} Portadas de sección para dividir capítulos
  \item 5 estilos de tablas profesionales con colores institucionales
  \item Recuadros especializados (resumen ejecutivo, datos clave, notas, definiciones)
  \item Sistema de citas y referencias en formato APA
  \item Tipografías institucionales (Patria y Noto Sans)
  \item Metadatos PDF para publicación web
\end{itemize}
\end{resumenejecutivo}

% ============================================================================
% CONTENIDO PRINCIPAL
% ============================================================================

\section{Presentación}

Este documento muestra todas las funcionalidades de la plantilla institucional SENER 2025. La plantilla ha sido actualizada para ofrecer una imagen más formal y estructurada, ideal para reportes oficiales y programas sectoriales.

\subsection{Características de la plantilla}

La plantilla incluye los siguientes elementos institucionales:

\begin{enumerate}
  \item Dos tipos de portada (estándar y con fondo guinda)
  \item Portadas de sección para organizar el contenido
  \item Página de créditos/directorio
  \item Estilos de títulos con colores institucionales
  \item Encabezados y pies de página personalizados
  \item Contraportada institucional
  \item Metadatos PDF embebidos
  \item Metadatos PDF embebidos
\end{enumerate}

% ============================================================================
% SECCIÓN 1: DISPOSICIONES DE TEXTO
% ============================================================================

\portadaseccion{1}{Disposiciones de Texto}{Ortotipografía y Columnas}

\section{Disposiciones de Texto y Ortotipografía}

Esta sección demuestra las capacidades de la plantilla para manejar diferentes disposiciones de texto.

\subsection{Texto a una columna}

El texto estándar se presenta a una columna, ideal para la lectura continua y documentos oficiales que requieren claridad y formalidad. \lipsum[1]

\subsection{Texto a dos columnas}

Para secciones que requieren mayor densidad de información o un estilo más periodístico, se puede utilizar el entorno de dos columnas:

\begin{multicols}{2}
\textbf{Columna 1:} \lipsum[2]

\columnbreak

\textbf{Columna 2:} \lipsum[3]
\end{multicols}

% ============================================================================
% SECCIÓN 1: ELEMENTOS DE LA PLANTILLA
% ============================================================================

\portadaseccion{2}{Elementos de la Plantilla}{Tipografía, Estilos y Recuadros}

\section{Tipografía y Texto}

La plantilla utiliza las tipografías institucionales \textbf{Patria} para títulos y \textbf{Noto Sans} para el cuerpo del texto, asegurando legibilidad y consistencia con la identidad gráfica del Gobierno de México.

\subsection{Recuadros y Cajas Destacadas}

Se han diseñado recuadros específicos para resaltar información clave:

\subsubsection{Recuadro informativo general}

\begin{recuadro}
Este es un recuadro informativo general. Úselo para destacar información relevante que complementa el texto principal sin interrumpir la lectura.
\end{recuadro}

\subsubsection{Nota importante}

\begin{notaimportante}
Las notas importantes utilizan el color guinda institucional. Son ideales para advertencias, requisitos legales o información crítica.
\end{notaimportante}

\subsubsection{Definiciones}

\begin{definicion}
\textbf{Sistema Eléctrico Nacional (SEN):} Conjunto de instalaciones destinadas a la generación, transmisión y distribución de energía eléctrica en todo el territorio nacional.
\end{definicion}

\subsubsection{Datos clave}

\begin{datosclave}
\textbf{Indicadores del sector eléctrico 2024:}
\begin{itemize}
  \item \textbf{Capacidad instalada:} \highlight{91,800 MW}
  \item \textbf{Demanda máxima:} \highlight{52,302 MW}
  \item \textbf{Energías limpias:} \highlight{31.2\%}
\end{itemize}
\end{datosclave}

\subsection{Citas Destacadas (Pull Quotes)}

Para resaltar frases impactantes o citas textuales importantes, utilice el entorno \texttt{destacado}:

\begin{destacado}[La planeación energética es fundamental para la soberanía nacional y el desarrollo sustentable de México.]
\end{destacado}

\subsection{Diagramas de Proceso}

Puede ilustrar procesos simples paso a paso utilizando el entorno \texttt{pasos}:

\begin{pasos}
  \paso{p1}{}{Diagnóstico}
  \paso{p2}{right of=p1}{Planeación}
  \paso{p3}{right of=p2}{Ejecución}
  
  \conectar{p1}{p2}
  \conectar{p2}{p3}
\end{pasos}

% ============================================================================
% SECCIÓN 2: TABLAS Y GRÁFICOS
% ============================================================================

\portadaseccion{3}{Tablas y Gráficos}{Visualización de Datos Institucionales}

\section{Tablas Profesionales}

Se ofrecen 5 estilos de tablas predefinidos para cubrir distintas necesidades de presentación de datos.

\subsection{Tabla estilo guinda (Datos Generales)}

\begin{tablaguinda}
  \caption{Capacidad instalada por región al cierre de 2024}
  \label{tab:capacidad-region}
  \begin{tabular}{lrrr}
    \toprule
    \rowcolor{gobmxGuinda}\encabezadoguinda{Región} & \encabezadoguinda{Capacidad (MW)} & \encabezadoguinda{Demanda (MW)} & \encabezadoguinda{Factor (\%)} \\
    \midrule
    Baja California & 3\,500 & 2\,300 & 68 \\
    Noroeste        & 5\,100 & 3\,900 & 73 \\
    Norte           & 6\,800 & 4\,500 & 71 \\
    Occidental      & 7\,200 & 5\,100 & 75 \\
    Central         & 12\,400 & 9\,800 & 79 \\
    \midrule
    \textbf{Total}  & \textbf{35\,000} & \textbf{25\,600} & \textbf{73} \\
    \bottomrule
  \end{tabular}
\end{tablaguinda}

\subsection{Tabla estilo verde (Sustentabilidad)}

\begin{tablaverde}
  \caption{Proyectos de energías renovables 2025-2030}
  \label{tab:renovables}
  \begin{tabular}{lrrr}
    \toprule
    \rowcolor{gobmxVerde}\encabezadoverde{Tecnología} & \encabezadoverde{Proyectos} & \encabezadoverde{Capacidad (MW)} & \encabezadoverde{Inversión (MDP)} \\
    \midrule
    Solar fotovoltaica & 45 & 5\,200 & 98\,000 \\
    Eólica & 28 & 3\,800 & 95\,000 \\
    Hidroeléctrica & 12 & 1\,200 & 48\,000 \\
    \midrule
    \textbf{Total} & \textbf{85} & \textbf{10\,200} & \textbf{241\,000} \\
    \bottomrule
  \end{tabular}
\end{tablaverde}

\subsection{Tabla estilo dorado (Finanzas)}

\begin{tabladorado}
  \caption{Inversión programada por sector 2025-2030 (MDP)}
  \label{tab:inversion}
  \begin{tabular}{lrr}
    \toprule
    \rowcolor{gobmxDorado}\encabezadodorado{Sector} & \encabezadodorado{Inversión} & \encabezadodorado{Participación (\%)} \\
    \midrule
    Generación & 850\,000 & 63.0 \\
    Transmisión & 320\,000 & 23.7 \\
    Distribución & 180\,000 & 13.3 \\
    \midrule
    \textbf{Total} & \textbf{1\,350\,000} & \textbf{100.0} \\
    \bottomrule
  \end{tabular}
\end{tabladorado}

\section{Ejemplos de Gráficos Institucionales}



% ============================================================================
% PRO TIP: CALIDAD DE IMÁGENES
% Para gráficos estadísticos (Excel, Python), use siempre formato PDF o EPS.
% Para fotografías, use JPG en alta resolución (300 dpi).
% Para capturas de pantalla, use PNG.
% EVITE usar imágenes pixeladas o de baja resolución.
% ============================================================================

La plantilla permite la inclusión de gráficos de alta resolución, mapas y diagramas complejos.

\subsection{Mapas a página completa}

\begin{figure}[H]
  \centering
  \includegraphics[width=1.0\textwidth]{img/graficos/mapa_sen_2025.png}
  \caption{Regiones y enlaces del Sistema Eléctrico Nacional en 2025. Detalle de la infraestructura de transmisión.}
  \label{fig:mapa-sen}
\end{figure}

\subsection{Gráficos de barras y tendencias}

\begin{figure}[H]
  \centering
  \includegraphics[width=1.0\textwidth]{img/graficos/adicion_capacidad.png}
  \caption{Adición de capacidad proyectada 2025-2030. Comparativa por tecnología y año.}
  \label{fig:adicion-capacidad}
\end{figure}

\subsection{Mapas de infraestructura}

\begin{figure}[H]
  \centering
  \includegraphics[width=1.0\textwidth]{img/graficos/mapa_gasoductos_2024.png}
  \caption{Red nacional de gasoductos en 2024. Infraestructura crítica para el sector energético.}
  \label{fig:mapa-gasoductos}
\end{figure}

\section{Figuras y Gráficos}

Las figuras deben incluir un pie de foto (caption) descriptivo. El estilo institucional coloca el caption debajo de la figura.

\begin{figure}[H]
  \centering
  \includegraphics[width=1.0\textwidth]{img/figura-.ejemplo.png}
  \caption{Regiones y enlaces del Sistema Eléctrico Nacional en 2025. El mapa muestra las nueve regiones de control y los principales enlaces de transmisión entre ellas.}
  \label{fig:sen2025}
\end{figure}

Como se observa en la Figura~\ref{fig:sen2025}, el Sistema Eléctrico Nacional está dividido en regiones interconectadas.

% ============================================================================
% SECCIÓN 3: REFERENCIAS Y ANEXOS
% ============================================================================

\portadaseccion{4}{Referencias y Anexos}{Información Complementaria}

\section{Sistema de Citas y Referencias}

La plantilla utiliza el formato \textbf{APA} para citas y referencias, gestionado por \texttt{biblatex}.

\subsection{Ejemplos de citas al pie}

La plantilla ahora soporta citas al pie de página para una lectura más fluida.

\begin{itemize}
    \item \textbf{Libros:} \textcite{rodriguez2023planeacion} analizan la planeación energética\footcite{rodriguez2023planeacion}.
    \item \textbf{Artículos:} El crecimiento de renovables es notable\footcite{gomez2023renovables}.
    \item \textbf{Reportes:} El \textcite{sener2024pladese} define la política sectorial\footcite{sener2024pladese}.
    \item \textbf{Sitios Web:} Consulte el portal oficial\footcite{sener2024portal}.
\end{itemize}

\section{Glosario de Términos}

\begin{description}
  \item[Capacidad instalada:] Potencia nominal de las centrales eléctricas disponibles para generar energía.
  \item[Factor de planta:] Relación entre la energía generada y la energía que se generaría operando a capacidad nominal.
  \item[Energías limpias:] Fuentes de energía que no emiten gases de efecto invernadero durante su operación.
  \item[Sistema Eléctrico Nacional (SEN):] Conjunto de instalaciones destinadas a la generación, transmisión y distribución de energía eléctrica.
\end{description}

% ============================================================================
% BIBLIOGRAFÍA
% ============================================================================

\clearpage
\printbibliography[title={Referencias Bibliográficas}]

% ============================================================================
% ANEXOS
% ============================================================================

\anexos

\section{Metodología de Cálculo}

Este anexo presenta la metodología utilizada para calcular las proyecciones de demanda energética.

\subsection{Modelo econométrico}

La ecuación general del modelo es:
\[
D_t = \beta_0 + \beta_1 PIB_t + \beta_2 POB_t + \beta_3 T_t + \varepsilon_t
\]
donde $D_t$ es la demanda de energía en el año $t$.

% ============================================================================
% CONTRAPORTADA
% ============================================================================

\contraportada{
  \textbf{Elaboración:}\\
  Dirección de Prospectiva del Sector Eléctrico\\
  Dirección de Análisis de Redes y Mercados Eléctricos\\
  Dirección de Planeación Energética\\[0.5cm]
  
  \textbf{Contacto:}\\
  Secretaría de Energía\\
  Insurgentes Sur 890, Col. Del Valle\\
  Ciudad de México, C.P. 03100\\
  Tel: (55) 5000-6000\\
  \url{www.gob.mx/sener}
}

\end{document}

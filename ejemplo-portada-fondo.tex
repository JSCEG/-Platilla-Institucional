%!TEX TS-program = xelatex
\documentclass{sener2025}

\usepackage{lipsum}

% ============================================================================
% INFORMACIÓN DEL DOCUMENTO
% ============================================================================
\title{Programa de Desarrollo del Sistema Eléctrico Nacional}
\subtitle{2025-2039}
\author{Unidad de Planeación Energética}
\date{Noviembre 2025}
\institucion{Secretaría de Energía}
\unidad{Unidad de Planeación Energética}
\setDocumentoCorto{PRODESEN 2025-2039}
\palabrasclave{energía, planeación, sistema eléctrico, PRODESEN}

\begin{document}

% ============================================================================
% PORTADA CON FONDO GUINDA
% ============================================================================
\portadafondo

% ============================================================================
% PÁGINA DE CRÉDITOS
% ============================================================================
\paginacreditos{
  \vspace{1cm}
  
  \textbf{Secretaria de Energía}\\
  Lic. Luz Elena González Escobar\\[0.5cm]
  
  \textbf{Subsecretario de Planeación y Transición Energética}\\
  Ing. Leonardo Beltrán Rodríguez\\[0.5cm]
  
  \textbf{Director General de Planeación e Información Energéticas}\\
  Dr. Edmundo Gil Borja\\[0.5cm]
  
  \textbf{Unidad de Planeación Energética}\\
  Equipo técnico de elaboración\\[1cm]
  
  \hrule
  \vspace{0.5cm}
  
  \textit{Primera edición: Noviembre 2025}\\
  \textit{Distribución gratuita. Prohibida su venta.}
}

% ============================================================================
% TABLA DE CONTENIDOS
% ============================================================================
\tableofcontents
\clearpage

% ============================================================================
% RESUMEN EJECUTIVO
% ============================================================================

\begin{resumenejecutivo}
El Programa de Desarrollo del Sistema Eléctrico Nacional (PRODESEN) 2025-2039 establece la estrategia de expansión y modernización del sector eléctrico mexicano para los próximos 15 años.

\textbf{Objetivos principales:}
\begin{itemize}
  \item Garantizar el suministro confiable de energía eléctrica
  \item Alcanzar el 50\% de generación con energías limpias para 2039
  \item Modernizar la infraestructura de transmisión y distribución
  \item Reducir las pérdidas técnicas y no técnicas del sistema
\end{itemize}

\textbf{Inversión proyectada:}
\begin{itemize}
  \item \textbf{Generación:} \highlight{\$850,000 MDP}
  \item \textbf{Transmisión:} \highlight{\$320,000 MDP}
  \item \textbf{Distribución:} \highlight{\$180,000 MDP}
  \item \textbf{Total:} \highlight{\$1,350,000 MDP}
\end{itemize}
\end{resumenejecutivo}

% ============================================================================
% CONTENIDO PRINCIPAL
% ============================================================================

\section{Introducción}

El Sistema Eléctrico Nacional (SEN) es fundamental para el desarrollo económico y social de México. Este programa establece las directrices para su crecimiento sostenible durante el periodo 2025-2039.

\subsection{Contexto del Sector Eléctrico}

\lipsum[1-2]

\section{Diagnóstico del Sistema Eléctrico}

\subsection{Capacidad Instalada Actual}

\begin{tablaguinda}
  \caption{Capacidad instalada por tecnología al cierre de 2024}
  \label{tab:capacidad-actual}
  \begin{tabular}{lrrr}
    \toprule
    \encabezadoguinda{Tecnología} & \encabezadoguinda{Capacidad (MW)} & \encabezadoguinda{Participación (\%)} & \encabezadoguinda{Factor de Planta (\%)} \\
    \midrule
    Ciclo combinado & 32,500 & 35.4 & 58.2 \\
    Hidroeléctrica & 12,600 & 13.7 & 42.5 \\
    Carboeléctrica & 5,400 & 5.9 & 65.8 \\
    Nuclear & 1,600 & 1.7 & 85.3 \\
    Eólica & 8,200 & 8.9 & 32.1 \\
    Solar & 7,800 & 8.5 & 24.5 \\
    Geotérmica & 950 & 1.0 & 78.4 \\
    Otras & 22,750 & 24.9 & 45.2 \\
    \midrule
    \textbf{Total} & \textbf{91,800} & \textbf{100.0} & \textbf{52.3} \\
    \bottomrule
  \end{tabular}
\end{tablaguinda}

\subsection{Demanda Eléctrica}

\begin{datosclave}
\textbf{Indicadores de demanda 2024:}
\begin{itemize}
  \item \textbf{Demanda máxima:} \highlight{52,302 MW}
  \item \textbf{Consumo anual:} \highlight{325,000 GWh}
  \item \textbf{Usuarios:} \highlight{47.2 millones}
  \item \textbf{Cobertura:} \highlight{99.2\%}
\end{itemize}
\end{datosclave}

\section{Proyecciones de Crecimiento}

\subsection{Escenarios de Demanda}

\begin{tablaverde}
  \caption{Proyección de demanda máxima por escenario (MW)}
  \label{tab:demanda-proyeccion}
  \begin{tabular}{lrrrr}
    \toprule
    \encabezadoverde{Año} & \encabezadoverde{Bajo} & \encabezadoverde{Medio} & \encabezadoverde{Alto} & \encabezadoverde{Crecimiento (\%)} \\
    \midrule
    2025 & 53,200 & 54,100 & 55,200 & 3.4 \\
    2027 & 55,800 & 57,500 & 59,800 & 6.3 \\
    2030 & 59,500 & 62,800 & 66,900 & 9.8 \\
    2035 & 66,200 & 71,500 & 78,200 & 13.9 \\
    2039 & 72,100 & 79,200 & 88,500 & 17.5 \\
    \bottomrule
  \end{tabular}
\end{tablaverde}

\section{Plan de Expansión}

\subsection{Proyectos de Generación}

\begin{tabladorado}
  \caption{Inversión en proyectos de generación 2025-2039 (MDP)}
  \label{tab:inversion-generacion}
  \begin{tabular}{lrrr}
    \toprule
    \encabezadodorado{Tecnología} & \encabezadodorado{Capacidad (MW)} & \encabezadodorado{Inversión} & \encabezadodorado{Proyectos} \\
    \midrule
    Solar fotovoltaica & 15,200 & 285,000 & 45 \\
    Eólica & 12,800 & 320,000 & 32 \\
    Hidroeléctrica & 2,400 & 95,000 & 8 \\
    Ciclo combinado & 8,500 & 120,000 & 12 \\
    Geotérmica & 850 & 30,000 & 5 \\
    \midrule
    \textbf{Total} & \textbf{39,750} & \textbf{850,000} & \textbf{102} \\
    \bottomrule
  \end{tabular}
\end{tabladorado}

\subsection{Modernización de Transmisión}

\lipsum[3]

\begin{notaimportante}
Los proyectos de transmisión requieren coordinación con las autoridades estatales y municipales para la obtención de permisos y derechos de vía.
\end{notaimportante}

\section{Metas de Energías Limpias}

\begin{tablagris}
  \caption{Metas de participación de energías limpias}
  \label{tab:metas-limpias}
  \begin{tabular}{lcccc}
    \toprule
    \encabezadogris{Año} & \encabezadogris{Capacidad Limpia (MW)} & \encabezadogris{Capacidad Total (MW)} & \encabezadogris{Participación (\%)} & \encabezadogris{Meta (\%)} \\
    \midrule
    2025 & 32,500 & 95,200 & 34.1 & 35.0 \\
    2030 & 48,200 & 110,800 & 43.5 & 40.0 \\
    2035 & 62,800 & 125,500 & 50.0 & 45.0 \\
    2039 & 75,400 & 138,200 & 54.6 & 50.0 \\
    \bottomrule
  \end{tabular}
\end{tablagris}

\section{Conclusiones}

El PRODESEN 2025-2039 establece una ruta clara para la transformación del sector eléctrico mexicano, priorizando la sostenibilidad, confiabilidad y accesibilidad del servicio.

\lipsum[4-5]

% ============================================================================
% CONTRAPORTADA
% ============================================================================

\contraportada{
  \textbf{Elaboración:}\\
  Dirección de Prospectiva del Sector Eléctrico\\
  Dirección de Análisis de Redes y Mercados Eléctricos\\
  Dirección de Planeación Energética\\[0.5cm]
  
  \textbf{Contacto:}\\
  Secretaría de Energía\\
  Insurgentes Sur 890, Col. Del Valle\\
  Ciudad de México, C.P. 03100\\
  Tel: (55) 5000-6000\\
  \url{www.gob.mx/sener}\\[0.5cm]
  
  \textbf{Disponible en:}\\
  \url{https://www.gob.mx/sener/documentos/prodesen}
}

\end{document}

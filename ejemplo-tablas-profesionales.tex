%!TEX TS-program = xelatex
\documentclass{sener2025}

\usepackage{lipsum}

% ============================================================================
% INFORMACIÓN DEL DOCUMENTO
% ============================================================================
\title{Ejemplo de Tablas Profesionales}
\subtitle{Guía de uso para documentos oficiales}
\author{Unidad de Planeación Energética}
\date{Noviembre 2025}
\institucion{Secretaría de Energía}
\unidad{Unidad de Planeación Energética}
\setDocumentoCorto{Ejemplo Tablas}
\palabrasclave{energía, tablas, documentos oficiales, SENER}
\version{1.0}

\begin{document}

\maketitle

\section{Tablas con Estilos Institucionales}

\subsection{Tabla estilo guinda (predeterminada)}

\begin{tablaguinda}
  \caption{Capacidad instalada por tecnología 2024}
  \label{tab:capacidad-tech}
  \begin{tabular}{lrrr}
    \toprule
    \encabezadoguinda{Tecnología} & \encabezadoguinda{Capacidad (MW)} & \encabezadoguinda{Participación (\%)} & \encabezadoguinda{Inversión (MDP)} \\
    \midrule
    Solar fotovoltaica & 8,500 & 15.2 & 85,000 \\
    Eólica & 7,200 & 12.9 & 72,000 \\
    Hidroeléctrica & 12,600 & 22.6 & 45,000 \\
    Ciclo combinado & 18,900 & 33.9 & 120,000 \\
    Carboeléctrica & 5,400 & 9.7 & 28,000 \\
    Nuclear & 1,600 & 2.9 & 65,000 \\
    Geotérmica & 950 & 1.7 & 12,000 \\
    Biomasa & 620 & 1.1 & 8,500 \\
    \midrule
    \textbf{Total} & \textbf{55,770} & \textbf{100.0} & \textbf{435,500} \\
    \bottomrule
  \end{tabular}
\end{tablaguinda}

\subsection{Tabla estilo verde}

\begin{tablaverde}
  \caption{Proyectos de energías renovables}
  \label{tab:renovables}
  \begin{tabular}{lrr}
    \toprule
    \encabezadoverde{Tecnología} & \encabezadoverde{Proyectos} & \encabezadoverde{Capacidad (MW)} \\
    \midrule
    Solar & 45 & 5,200 \\
    Eólica & 28 & 3,800 \\
    Hidroeléctrica & 12 & 1,200 \\
    Geotérmica & 5 & 450 \\
    Biomasa & 8 & 320 \\
    \midrule
    \textbf{Total} & \textbf{98} & \textbf{10,970} \\
    \bottomrule
  \end{tabular}
\end{tablaverde}

\subsection{Tabla estilo dorado}

\begin{tabladorado}
  \caption{Inversión por región 2024-2030}
  \label{tab:inversion}
  \begin{tabular}{lrr}
    \toprule
    \encabezadodorado{Región} & \encabezadodorado{Inversión (MDP)} & \encabezadodorado{Proyectos} \\
    \midrule
    Norte & 125,000 & 18 \\
    Centro & 185,000 & 25 \\
    Sur & 95,000 & 15 \\
    Occidente & 110,000 & 12 \\
    Oriente & 145,000 & 20 \\
    \midrule
    \textbf{Total} & \textbf{660,000} & \textbf{90} \\
    \bottomrule
  \end{tabular}
\end{tabladorado}

\subsection{Tabla estilo gris (neutral)}

\begin{tablagris}
  \caption{Indicadores técnicos del sistema}
  \label{tab:tecnicos}
  \begin{tabular}{lcccc}
    \toprule
    \encabezadogris{Indicador} & \encabezadogris{2021} & \encabezadogris{2022} & \encabezadogris{2023} & \encabezadogris{2024} \\
    \midrule
    Confiabilidad (\%) & 99.1 & 99.3 & 99.4 & 99.5 \\
    Pérdidas (\%) & 14.2 & 13.8 & 13.5 & 13.1 \\
    Factor de planta (\%) & 52.3 & 53.1 & 54.2 & 55.0 \\
    \bottomrule
  \end{tabular}
\end{tablagris}

\subsection{Tabla limpia (sin stripes)}

\begin{tablalimpia}
  \caption{Proyectos prioritarios del sector eléctrico}
  \label{tab:proyectos}
  \begin{tabularx}{\textwidth}{lXrr}
    \toprule
    \encabezadoguinda{Clave} & \encabezadoguinda{Proyecto} & \encabezadoguinda{Capacidad (MW)} & \encabezadoguinda{Año} \\
    \midrule
    PE-001 & Central Solar Villanueva Fase III & 828 & 2026 \\
    PE-002 & Parque Eólico Istmo de Tehuantepec & 650 & 2026 \\
    PE-003 & Modernización Hidroeléctrica Malpaso & 450 & 2027 \\
    PE-004 & Ciclo Combinado Norte III & 1,200 & 2027 \\
    PE-005 & Interconexión Baja California Sur & 300 & 2028 \\
    \bottomrule
  \end{tabularx}
\end{tablalimpia}

\clearpage

\section{Combinación de Estilos}

\subsection{Tabla verde con encabezado dorado}

\begin{tablaverde}
  \caption{Mezcla de estilos - encabezado dorado con stripes verdes}
  \label{tab:mezcla1}
  \begin{tabular}{lrr}
    \toprule
    \encabezadodorado{Concepto} & \encabezadodorado{Valor 2023} & \encabezadodorado{Valor 2024} \\
    \midrule
    Generación limpia (GWh) & 98,500 & 112,300 \\
    Emisiones evitadas (ton CO₂) & 45,200 & 51,800 \\
    Inversión verde (MDP) & 85,000 & 95,000 \\
    \bottomrule
  \end{tabular}
\end{tablaverde}

\subsection{Tabla dorada con encabezado verde}

\begin{tabladorado}
  \caption{Otra combinación - encabezado verde con stripes dorados}
  \label{tab:mezcla2}
  \begin{tabular}{lcc}
    \toprule
    \encabezadoverde{Programa} & \encabezadoverde{Beneficiarios} & \encabezadoverde{Presupuesto (MDP)} \\
    \midrule
    Electrificación rural & 125,000 & 450 \\
    Eficiencia energética & 85,000 & 320 \\
    Energía comunitaria & 45,000 & 180 \\
    \bottomrule
  \end{tabular}
\end{tabladorado}

\section{Recuadros para Documentos Oficiales}

\subsection{Resumen Ejecutivo}

\begin{resumenejecutivo}
El sector eléctrico nacional presenta un crecimiento sostenido en la incorporación de energías limpias, alcanzando el \highlight{35\%} de la capacidad instalada total en 2024.

\textbf{Principales logros:}
\begin{itemize}
  \item Incremento del 12\% en capacidad solar
  \item Reducción del 8\% en emisiones de CO₂
  \item Inversión de \$435,500 MDP en infraestructura
\end{itemize}

\textbf{Retos identificados:}
\begin{itemize}
  \item Modernización de redes de transmisión
  \item Almacenamiento de energía a gran escala
  \item Integración de sistemas aislados
\end{itemize}
\end{resumenejecutivo}

\subsection{Datos Clave del Sector}

\begin{datosclave}
\begin{itemize}
  \item \textbf{Capacidad instalada total:} \highlight{55,770 MW}
  \item \textbf{Generación anual:} \highlight{325,000 GWh}
  \item \textbf{Usuarios atendidos:} \highlight{47.2 millones}
  \item \textbf{Cobertura nacional:} \highlight{99.2\%}
  \item \textbf{Inversión 2024:} \highlight{\$435,500 MDP}
\end{itemize}
\end{datosclave}

\section{Combinación de Elementos}

\lipsum[1]

\begin{notaimportante}
Los datos presentados en este documento son preliminares y están sujetos a validación por parte de las áreas técnicas correspondientes.
\end{notaimportante}

\begin{tablaguinda}
  \caption{Indicadores de desempeño del sector eléctrico}
  \label{tab:indicadores}
  \begin{tabular}{lcccc}
    \toprule
    \encabezadoguinda{Indicador} & \encabezadoguinda{2021} & \encabezadoguinda{2022} & \encabezadoguinda{2023} & \encabezadoguinda{2024} \\
    \midrule
    Confiabilidad (\%) & 99.1 & 99.3 & 99.4 & 99.5 \\
    Pérdidas técnicas (\%) & 14.2 & 13.8 & 13.5 & 13.1 \\
    Factor de planta (\%) & 52.3 & 53.1 & 54.2 & 55.0 \\
    Tiempo de interrupción (min) & 185 & 172 & 165 & 158 \\
    \bottomrule
  \end{tabular}
\end{tablaguinda}

\lipsum[2]

\end{document}

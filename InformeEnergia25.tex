\documentclass{sener2025}

\addbibresource{referencias.bib}

% --- Metadatos del Documento ---
\title{Informe Institucional de Energía 2025}
\subtitle{Avances, retos y perspectivas de la transición energética en México}
\author{Dirección General de Planeación Energética}
\date{30 de junio de 2025}
\institucion{Secretaría de Energía (SENER)}
\unidad{Unidad de Planeación y Transición Energética}
\setDocumentoCorto{InformeEnergia25}
\palabrasclave{transición energética; eficiencia; renovables; seguridad energética}
\version{1}

\begin{document}

\portadafondo

\tableofcontents
\newpage

\listafiguras
\newpage

\listatablas
\newpage

\begin{resumenejecutivo}
Este informe institucional presenta un panorama integrado del sector energético mexicano al cierre de 2025, destacando el avance en la incorporación de energías limpias, la modernización de la infraestructura y los esfuerzos regulatorios para fortalecer la seguridad energética. Se describen los principales indicadores de generación, consumo y emisiones, así como la evolución de los marcos normativos y de planeación sectorial.
 
 El documento también analiza los retos pendientes en materia de confiabilidad del sistema eléctrico, diversificación de la matriz, acceso equitativo a la energía y desarrollo de infraestructura estratégica. Finalmente, se presentan recomendaciones de política pública y líneas de acción coordinadas entre dependencias para consolidar la transición energética con criterios de sostenibilidad, competitividad y justicia social.
\end{resumenejecutivo}

\begin{datosclave}
  \begin{itemize}
    \item Capacidad renovable instalada superó el 35\% de la capacidad total
    \item Intensidad energética nacional se redujo 8\% respecto a 2020
    \item Emisiones del sector eléctrico disminuyeron 12\% frente al escenario tendencial.
  \end{itemize}
\end{datosclave}

\section{Contexto general del sistema energético mexicano}

El sistema energético mexicano enfrenta un proceso de transformación profunda impulsado por la transición energética, los compromisos internacionales de mitigación de emisiones y la necesidad de fortalecer la seguridad de suministro. En 2025, la matriz energética sigue dependiendo de los hidrocarburos, pero con una participación creciente de las energías renovables y de medidas de eficiencia en todos los eslabones de la cadena de valor.\footnote{ Los datos presentados en esta sección se basan en el Balance Nacional de Energía 2024.}



\section{Evolución de la capacidad de generación eléctrica}

Durante el periodo 2020–2025, la capacidad de generación eléctrica instalada en México presentó un crecimiento moderado, con una recomposición hacia tecnologías de menor intensidad de emisiones. Las centrales de ciclo combinado a gas natural mantuvieron una participación relevante, mientras que la capacidad eólica y fotovoltaica experimentó tasas de crecimiento superiores al promedio del sistema.\cite{bne2024_generacion}
\begin{itemize}
  \item Incremento de capacidad fotovoltaica distribuida en zonas urbanas;
  \item Mayor participación de parques eólicos en regiones con alto recurso de viento;
  \item Sustitución gradual de centrales de combustóleo y carbón.
\end{itemize}
\begin{figure}[H]
  \centering
  \includegraphics[width=0.8\textwidth]{img/graficos/figura_2_1.png}
  \caption{Evolución de la capacidad instalada por tecnología, 2020–2025}
\end{figure}
\fuente{Elaboración propia con datos del Balance Nacional de Energía 2024 (SENER).}

\begin{tabladorado}
  \caption{Capacidad instalada de generación eléctrica por tecnología seleccionada}
  \label{tab:capacidad_instalada_de_generac}
  \begin{tabular}{lcccc}
    \toprule
    \rowcolor{gobmxDorado} \encabezadodorado{A} & \encabezadodorado{B} & \encabezadodorado{C} & \encabezadodorado{D} & \encabezadodorado{E} \\
    \midrule
    Generación & 1 & 2 & 3 & 4 \\
    Distribucion & 56 & 6 & 7 & 8 \\
    Total & 57 & 8 & 10 & 12 \\
    \bottomrule
  \end{tabular}
\end{tabladorado}
\fuente{Elaboración propia con base en el Balance Nacional de Energía}



\subsection{Integración de energías renovables variables al sistema eléctrico}

La integración de fuentes renovables variables (eólica y solar) plantea retos técnicos para la operación del sistema eléctrico, especialmente en términos de flexibilidad, reservas operativas y capacidad de transmisión. Para enfrentar estos desafíos, se requieren inversiones en redes, sistemas de almacenamiento y capacidades de respuesta de la demanda.\footnote{ La información técnica de esta subsección se apoya en estudios de planeación del sistema eléctrico.} \cite{cenace2023_flexibilidad}


\subsubsection{Papel de las Energías renovables}

Puro texto de idea


\paragraph{Papel de las Energías renovables Sub}

Puro texto de idea


\section{Consumo final de energía por sector}

El consumo final de energía en México continúa concentrándose en los sectores transporte, industrial y residencial. El transporte depende mayoritariamente de combustibles fósiles, mientras que en el sector industrial se observa una adopción gradual de procesos más eficientes y de combustibles más limpios. En el sector residencial, las políticas de eficiencia energética han promovido el uso de equipos con menor consumo específico.\begin{destacado}
 La reducción de la intensidad energética es un factor clave para desacoplar el crecimiento económico del aumento en el consumo de energía.
\end{destacado}
\begin{itemize}
  \item Transporte: predominio de gasolinas y diésel;
  \item Industria: uso creciente de gas natural y cogeneración eficiente;
  \item Residencial: mayor participación de electricidad y gas LP como energéticos principales.
\end{itemize}
\begin{figure}[H]
  \centering
  \includegraphics[width=0.8\textwidth]{img/graficos/figura_2_12.png}
  \caption{Distribución del consumo final de energía por sector, 2024}
\end{figure}
\fuente{Cálculos de la Unidad de Planeación con información de SIE–SENER.}

\begin{tabladorado}
  \caption{Consumo final de energía por sector económico}
  \label{tab:consumo_final_de_energia_por_s}
  \begin{tabular}{lccccccccccc}
    \toprule
    \rowcolor{gobmxDorado} \encabezadodorado{TECNOLOGÍA} & \encabezadodorado{2014 1/} & \encabezadodorado{2015 1/} & \encabezadodorado{2016 1/,7/} & \encabezadodorado{2017 1/,11/} & \encabezadodorado{2018 1/,10/} & \encabezadodorado{2019 1/,10/} & \encabezadodorado{2020 1/,5/} & \encabezadodorado{2021 1/,5/} & \encabezadodorado{2022 1/,5/} & \encabezadodorado{2023 1/,5/} & \encabezadodorado{2024 6/} \\
    \midrule
    Hidroeléctrica & 12551.774000000005 & 12560.174000000003 & 12589 & 12612 & 12612 & 12611.785499999998 & 12611.785500000002 & 12613.9855 & 12613.080000000002 & 12611.93 & 12611.93 \\
    Geotermoeléctrica & 873.6 & 898.6 & 909 & 898.6 & 898.6 & 898.6 & 950.6 & 975.6 & 975.6 & 975.6 & 975.6 \\
    Eoloeléctrica & 2659.5 & 2877 & 3735 & 3898 & 4865.965 & 6050.465 & 6504.165 & 6977.165 & 6921.325 & 7055.325 & 7512.165 \\
    Fotovoltaica & 55.41408 & 57.23568 & 145 & 170.81107999999998 & 1877.6340799999998 & 3646.3640800000003 & 5149.270079999999 & 5954.60608 & 6515.33568 & 7437.172600000001 & 7960.943000000001 \\
    Bioenergía 2/ & 233.21599999999998 & 233.21599999999998 & 889 & 374.15 & 375.15 & 375.15 & 377.95 & 377.95 & 407.826 & 406.777 & 387.342 \\
    Suma limpia renovable & 16373.504080000006 & 16626.225680000003 & 18267 & 17953.56108 & 20629.349080000004 & 23582.36458 & 25593.77058 & 26899.30658 & 27433.166680000002 & 28486.8046 & 29447.98 \\
    Nucleoeléctrica & 1400 & 1510 & 1608 & 1608 & 1608 & 1608 & 1608 & 1608 & 1608 & 1608 & 1608 \\
    Cogeneración Eficiente & 819.3149999999999 & 943.266 & 1036 & 1321.8129999999999 & 1708.876 & 1709.875 & 2304.6870000000004 & 2304.6870000000004 & 2307.7000000000003 & 2322.3260000000005 & 2293.273 \\
    Frenos Regenerativos & 7 & 6.5 & 6.61 &  &  &  &  &  &  &  &  \\
    Generación Distribuida (GD) 8/ &  &  & 248 &  &  &  &  &  &  &  &  \\
    FIRCO 9/ & 0.3 & 12.5 & 13.6 &  &  &  &  &  &  &  &  \\
    Hibrido Baterías-FV Solar &  &  &  &  &  &  &  &  & 20 & 32 & 92 \\
    Suma limpia no renovable & 2226.6150000000002 & 2472.266 & 2912.21 & 2929.813 & 3316.876 & 3317.875 & 3912.6870000000004 & 3912.6870000000004 & 3935.7000000000003 & 3962.3260000000005 & 3993.273 \\
    Total energía limpia & 18600.119080000008 & 19098.491680000003 & 21179.21 & 20883.37408 & 23946.225080000004 & 26900.23958 & 29506.457580000002 & 30811.993580000002 & 31368.866680000003 & 32449.1306 & 33441.253 \\
    Porcentaje & 0.012506597230038842 & 0.4387914069899119 & 0.26920031670625494 & 0 & 0 & 0 & 0 & 0 & 0 & 0 & 0 \\
    Ciclo combinado & 22698.547 & 22948.547 & 27274 & 25340.047 & 27393.451999999997 & 30402.08 & 31947.604 & 33640.367999999995 & 34412.569 & 35178.329 & 35668.929 \\
    Térmica convencional 3/ & 12664.95 & 12664.95 & 13174 & 12664.95 & 12314.95 & 11830.95 & 11808.95 & 11792.95 & 11342.95 & 11300.05 & 11300.05 \\
    Turbogás 4/ & 2398.7340000000004 & 2848.7340000000004 & 5052 & 2959.814 & 2959.814 & 2959.814 & 3545.014 & 3743.594 & 3814.5940000000005 & 3887.714 & 3952.57 \\
    Combustión interna & 539.946 & 539.946 & 1453 & 738.6530000000001 & 879.655 & 890.6080000000001 & 849.5120000000001 & 700.6020000000001 & 727.581 & 729.4350000000001 & 716.515 \\
    Carboeléctrica & 5463.450000000001 & 5463.450000000001 & 5378 & 5463.450000000001 & 5463.450000000001 & 5463.450000000001 & 5463.450000000001 & 5463.450000000001 & 5463.450000000001 & 5463.450000000001 & 5463.45 \\
    TOTAL & 62365.74608000001 & 63564.118680000014 & 73510.20999999999 & 68050.28808 & 72957.54608 & 78447.14158 & 83120.98758000002 & 86152.95757999999 & 87130.01068 & 89008.1086 & 90542.76699999999 \\
    \bottomrule
  \end{tabular}
\end{tabladorado}
\fuente{Cálculos de la Unidad de Planeación con datos de SIE–SENER}



\section{Emisiones de gases de efecto invernadero del sector energético}

El sector energético es responsable de la mayor proporción de las emisiones nacionales de gases de efecto invernadero. Entre 2020 y 2025, se ha observado una reducción gradual de las emisiones asociadas a la generación eléctrica, resultado de la modernización de centrales, la sustitución de combustibles y la mayor participación de energías limpias.\cite{inventarioGEI2024} \footnote{ Las cifras de emisiones se presentan utilizando el potencial de calentamiento global a 100 años (GWP100).}
\begin{figure}[H]
  \centering
  \includegraphics[width=0.8\textwidth]{img/graficos/figura_3_5.png}
  \caption{Emisiones de GEI del sector energético y contribución de la generación eléctrica}
\end{figure}
\fuente{Inventario Nacional de Emisiones de GEI 2024 y proyecciones institucionales.}



\subsection{Medidas de mitigación y escenarios de transición}

Los escenarios de transición energética consideran combinaciones de medidas de eficiencia, electrificación de usos finales, penetración de energías renovables y uso de nuevas tecnologías como el almacenamiento y el hidrógeno de bajas emisiones. Estas medidas permiten trazar rutas costo-eficientes para alcanzar los objetivos de reducción de emisiones.\begin{destacado}
 La planeación integrada de energía y clima permite identificar trayectorias de transición que maximizan los beneficios económicos, ambientales y sociales.
\end{destacado}


\anexos

\section{Metodología de Cálculo}

Este anexo describe...


\subsection{Modelo Econométrico}

La ecuación general del modelo es:

 \begin{equation}
 D_t = \beta_0 + \beta_1 PIB_t + \beta_2 POB_t + \beta_3 T_t + \varepsilon_t
\end{equation}

 donde $ D_t$ es la demanda de energía en el año $ t$.


\subsection{Ejemplo de Bloques}

La plantilla ahora soporta citas al pie...

\begin{ejemplo}[title={Ejemplo}]
\begin{itemize}
  \item Libros: Rodríguez y García analizan...
  \item Artículos: El crecimiento es notable.
  \item Reportes: La Secretaría define...
\end{itemize}

\end{ejemplo}


\portadaseccion{1}{Referencias y Anexos}{Información Complementaria}

\section{Sistema de Citas}

...


\paginacreditos{
\begin{center}
{\patriafont\fontsize{12}{14}\selectfont\color{gobmxGuinda} Mtra. Luz Elena González Escobar}\\
{\patriafont\fontsize{9}{11}\selectfont Secretaria de Energía}\\[0.5cm]
{\patriafont\fontsize{12}{14}\selectfont\color{gobmxGuinda} Mtro. Juan José Vidal Amaro}\\
{\patriafont\fontsize{9}{11}\selectfont Subsecretario de Hidrocarburos}\\[0.5cm]
\end{center}
}
\section*{Glosario}
\addcontentsline{toc}{section}{Glosario}

\entradaGlosario{Energías Limpias}{Fuentes de energía que no emite}

\section*{Siglas y Acrónimos}
\addcontentsline{toc}{section}{Siglas y Acrónimos}

\entradaSigla{CENACE}{Centro Nacional de Control de Energía}

\printbibliography

\contraportada{
\textbf{\textcolor{gobmxDorado}{ Elaboración:}}\\
Dirección de Prospectiva del Sector Eléctrico\\
Dirección de Análisis de Redes y Mercados Eléctricos\\
Dirección de Planeación Energética\\[0.5cm]
\textbf{\textcolor{gobmxDorado}{ Contacto:}}\\
Secretaría de Energía\\
Jalapa 20, Col. Roma Norte\\
Ciudad de México, C.P. 06700\\
Tel: (55) 5000-6000\\
\textbf{\textcolor{gobmxGuinda}{ www.gob.mx/sener}}
}

\end{document}

\documentclass{sener2025}

\title{PRODESEN 2025-2030}
\author{Dr. Jorge Marcial Islas Samperio}
\date{noviembre 2025}

\begin{document}

\portadafondo

\tableofcontents
\clearpage

\portadaseccion{1}{Disposiciones de Texto}{Ortotipografía y Columnas}

\section{Disposiciones de Texto}

\subsection{Texto a una columna}

El texto estándar se presenta a una columna, ideal para la lectura continua y documentos oficiales que requieren claridad y formalidad.

\subsection{Texto a dos columnas}

Para secciones que requieren mayor densidad de información o un estilo más periodístico, se puede utilizar el entorno de dos columnas. Este formato es especialmente útil para comparaciones o listados extensos.

\portadaseccion{2}{Elementos de la Plantilla}{Tipografía y Estilos}

\section{Elementos de la Plantilla}

\subsection{Tipografía y Texto}

La plantilla utiliza las tipografías institucionales **Patria** para títulos y **Noto Sans** para el cuerpo del texto, asegurando legibilidad y consistencia con la identidad gráfica del Gobierno de México.

\subsection{Recuadros y Cajas Destacadas}

Se han diseñado recuadros específicos para resaltar información clave en diferentes contextos.

\subsection{Citas Destacadas}

\portadaseccion{3}{Tablas y Gráficos}{Visualización de Datos Institucionales}

\section{Tablas y Gráficos}

\subsection{Tablas Profesionales}

Se ofrecen 5 estilos de tablas predefinidos para cubrir distintas necesidades de presentación de datos.

\subsection{Ejemplos de Gráficos Institucionales}

La plantilla permite la inclusión de gráficos de alta resolución, mapas y diagramas complejos.

\begin{tablaguinda}
  \begin{tabular}{llll}
\toprule
Región & Capacidad (MW) & Demanda (MW) & Factor (%) \\
\midrule
Baja California & 3500 & 2300 & 68 \\
Noroeste & 5100 & 3900 & 73 \\
Norte & 6800 & 4500 & 71 \\
Occidental & 7200 & 5100 & 75 \\
Central & 12400 & 9800 & 79 \\
Total & 35000 & 25600 & 73 \\
\bottomrule
  \end{tabular}
  \caption{Capacidad instalada por región al cierre de 2024}
\end{tablaguinda}

\begin{tablaverde}
  \begin{tabular}{llll}
\toprule
Tecnología & Proyectos & Capacidad (MW) & Inversión (MDP) \\
\midrule
Solar fotovoltaica & 45 & 5200 & 98000 \\
Eólica & 28 & 3800 & 95000 \\
Hidroeléctrica & 12 & 1200 & 48000 \\
Total & 85 & 10200 & 241000 \\
\bottomrule
  \end{tabular}
  \caption{Proyectos de energías renovables 2025-2030}
\end{tablaverde}

\begin{tabladorado}
  \begin{tabular}{lll}
\toprule
Sector & Inversión & Participación (%) \\
\midrule
Generación & 850000 & 63 \\
Transmisión & 320000 & 23.7 \\
Distribución & 180000 & 13.3 \\
Total & 1350000 & 100 \\
\bottomrule
  \end{tabular}
  \caption{Inversión programada por sector 2025-2030 (MDP)}
\end{tabladorado}

\portadaseccion{4}{Elementos Avanzados}{Funcionalidades de Alto Nivel}

\section{Elementos Avanzados}

de la funcionalidad de **Letra Capital** (Drop Cap). Este estilo es común en publicaciones editoriales de alta calidad y ayuda a guiar la vista del lector al inicio de una sección importante.

\subsection{Códigos QR Generados}

Para documentos impresos que requieren enlazar a recursos digitales, la plantilla puede generar códigos QR automáticamente.

\subsection{Badges y Etiquetas}

Los badges son perfectos para categorizar o destacar información clave de forma visual.

\subsection{Barras de Progreso}

Las barras de progreso son ideales para visualizar avances hacia metas energéticas y objetivos institucionales.

\subsection{Líneas de Tiempo}

Las timelines permiten visualizar cronologías de proyectos o hitos históricos del sector energético.

\portadaseccion{5}{Referencias y Anexos}{Información Complementaria}

\section{Referencias y Anexos}

\printbibliography

\subsection{Sistema de Citas y Referencias}

La plantilla utiliza el formato **APA** para citas y referencias, gestionado por biblatex.

\subsection{Glosario de Términos}

\end{document}